\chapter{Introduzione}

Negli ultimi anni i dispositivi indossabili per la registrazione di video in prima persona hanno conosciuto una diffusione sempre più ampia. Strumenti come \emph{smart glasses}, \emph{body cameras}, \emph{action cameras} hanno reso possibile la cattura di flussi video continui dal punto di vista dell'utilizzatore, dando origine a quello che viene comunemente definito come \emph{egocentric vision}. Ciò che li rende peculiari è la loro capacità di catturare dettagli e prospettive uniche, fornendo una visione diretta dell'attività di chi li indossa.

\begin{figure}[ht]
    \centering
    \begin{minipage}{0.3\linewidth}
        \centering
        \includegraphics[width=\linewidth]{Images/hololens.png}\\
        Microsoft HoloLens \cite{MicrosoftHoloLens2025}
    \end{minipage}
    \hfill
    \begin{minipage}{0.3\linewidth}
        \centering
        \includegraphics[width=\linewidth]{Images/applepro.jpg}\\
        Apple Vision Pro \cite{AppleVisionPro2025}
    \end{minipage}
    \hfill
    \begin{minipage}{0.3\linewidth}
        \centering
        \includegraphics[width=\linewidth]{Images/ariaglasses.png}\\
        Meta Aria Glasses \cite{ProjectAria2025}
    \end{minipage}
    \caption{Dispositivi indossabili per la cattura di video egocentrici}
    \label{fig:dispositivi_egocentrici}
\end{figure}

L'adozione crescente di tali dispositivi è stata favorita dalla loro versatilità: da un lato vengono utilizzati per scopi ricreativi e per la condivisione di esperienze personali, dall'altro trovano applicazione in contesti professionali e industriali, dove consentono di documentare procedure complesse e migliorare i processi di formazione e supervisione.

Il principale ostacolo all'analisi di questi contenuti risiede nella loro natura non strutturata. I video in prima persona possono essere considerati come veri e propri flussi di coscienza visivi: lunghi, frammentati, privi di un'organizzazione narrativa chiara e difficili da interpretare. La presenza di movimenti rapidi della fotocamera, variazioni di illuminazione e interazioni simultanee con più oggetti rende complicata l'estrazione di significato. Un'annotazione manuale completa non è praticabile, sia per la mole di dati prodotta sia per la complessità dei contenuti.

Da questa problematica emerge la necessità di costruire una “memoria artificiale” in grado di trasformare i video egocentrici in una rappresentazione più organizzata e interpretabile. In questa tesi analizzeremo innanzitutto i principali contributi presenti in letteratura, per poi concentrarci su \textbf{AMEGO}\cite{goletto2024amego}, un sistema sviluppato per strutturare e rendere interrogabili i video egocentrici e attualmente considerato stato dell'arte nel suo ambito \cite{goletto2024amego}.

Durante la fase sperimentale, valuteremo \emph{AMEGO} in un contesto diverso da quello in cui era stato testato. Nel lavoro originale è stato valutato sul dataset \emph{EPIC KITCHENS} \cite{Damen2021PAMI}, costituito da video ambientati in cucine domestiche. L'analisi si concentrerà sul dataset \textbf{ENIGMA-51} \cite{ragusa2023enigma51}, acquisito in contesti industriali, in cui diversi operatori hanno seguito procedure guidate per eseguire attività di riparazione di quadri elettrici. Andremo quindi a strutturare le annotazioni presenti nel nuovo dataset secondo il formato proposto da AMEGO per poi valutare i risultati calcolando l'accuracy \footnote{\textbf{Accuracy}: proporzione di risposte corrette rispetto al totale delle risposte fornite.} totale. La differenza tra i due domini rende lo studio particolarmente interessante, in quanto consente di valutare la capacità del modello di generalizzare a contesti applicativi mai visti.

In ambito industriale, l'analisi dei video egocentrici è fondamentale per ottimizzare vari aspetti operativi. Garantire ad esempio che le operazioni vengano eseguite nell'ordine corretto migliora i flussi produttivi e riduce il rischio di errori umani. Inoltre, la capacità di riconoscere quali strumenti vengono utilizzati contemporaneamente ad altri oggetti, soprattutto se potenzialmente pericolosi, contribuisce a migliorare la sicurezza dei lavoratori. Quest'ultimo tema prende il nome di \textbf{concurrency}. Sarà di centrale importanza nella fase sperimentale, poiché la valutazione si baserà su domande del tipo:
\begin{quote}
    ``Con quali oggetti ho interagito \textbf{simultaneamente}''
\end{quote}
Come verrà mostrato nel Cap.~\ref{chapter:Risultati}, i risultati mostreranno che il livello di dettaglio richiesto da \emph{ENIGMA} costituisce attualmente una sfida significativa nell'ambito \texttt{semantic-free}.